\documentclass{article}

\usepackage{listings}
\usepackage{color}

\definecolor{dkgreen}{rgb}{0,0.6,0}
\definecolor{gray}{rgb}{0.5,0.5,0.5}
\definecolor{mauve}{rgb}{0.58,0,0.82}

\lstset{frame=single,
  language=VBScript,
  aboveskip=3mm,
  belowskip=3mm,
  captionpos=b,
  showstringspaces=false,
  columns=flexible,
  basicstyle={\small\ttfamily},
  numbers=none,
  numberstyle=\tiny\color{gray},
  keywordstyle=\color{blue},
  commentstyle=\color{dkgreen},
  stringstyle=\color{mauve},
  breaklines=true,
  breakatwhitespace=true,
  tabsize=2
}


\title{Master Thesis}
\author{Ulrika Malmgren}

\begin{document}
	\pagenumbering{gobble}
	\maketitle
	\newpage


	\section{Abstract}
	This is where I'll write my abstract

	\tableofcontents
	\newpage

	\pagenumbering{arabic}

	\section{Introduction}
	\subsection{Background}
		\subsubsection{About Nobina}
		Nobina is the Nordic leader in scheduled public road transport services, with a market share of 16\%.
		The company is also one of the ten largest public transport companies in Europe. It's most wellknown brand is Swebus Express which 2 million passengers use each year for long distance
		traveling. Nobina has around 280 million passengers per year within the company’s two business units: contract traffic and express bus traffic. The bus fleet includes 3,500 vehicles and the company has 7,500 employees.

		\subsubsection{About the OMS system}
		As support for the production of these transport services, Nobina has built a specialized system, the Operation Management System (OMS) which was first released in 2003. OMS is comprised of 15 modules within 3 different areas: staff planning, vehicle maintenance and traffic management. It is unique since there is no system on the market that combines these areas. The system has 900 active users and is used in four countries (Sweden, Finland, Norway and Denmark).

		OMS is a Winform application written in Vb.Net (80 \%) and C\# (20 \%). There is a general plan for the architecture of the modules but since each developer is responsible for one or two modules different programming styles exist.

		This thesis will look more closely at the Workshop and Fuel modules in OMS. In the Workshop module, there are functions related to maintaining the vehicles in good condition such as error reports, servicing, workshop planning, and so on. Fuel on the other hand, handles the fuel consumption of the vehicles, the amount of kilometers driven, the fuel levels at the different garages, etc.

		\subsubsection{Releases and testing} 
		The system is released four times per year and warmfixes are released about once a month per module or two times a week for the entire system. Before each release manual testing is performed. 

		For testing a release, 20 people are bought in externally and test the system for two days. Critical bugs are fixed and the system retested during an additional day.
		
		For a warmfix, only the concerned functionality is tested but for a regular release the entire system is tested.

	\subsection{Problem statement}
		Testing in this way is really expensive and Nobina wishes to lower their testing costs and using automated testing instead. 

	\section{Testing}
		\subsection{Purpose}
		\subsection{Different levels of testing}
		\subsection{Prerequisites}
		What makes for a good test? Reproducable -> problems with test environment are interesting - test data?

	\section{GUI testing}
		\subsection{Introduction}
		A black box testing technique for end-to-end tests. 
		\subsection{Advantages}
		\subsection{Drawbacks}
		Time consuming, false positives, flaky

	\section{Unit testing}
		\subsection{Introduction}
		A white box testing technique for small units of functionality
		\subsection{Advantages}
		\subsection{Drawbacks}

	\section{Legacy code and refactoring}
		\subsection{Introduction}
		The OMS system has been running for seven years already at the time of the thesis and is considered to have legacy code. Before unit testing can be used to test the system, it needs to be refactored. 

		\subsection{Legacy code}
		
		\subsection{Refactoring}
		Refactoring is the process of changing a software system in such a way that it does not alter the behavior of the code yet improves its internal structure and a refactoring is a change made to the internal structure of software to make it easier to understand and cheaper to modify without changing its observable behavior. \cite{refactor}

			\subsubsection{Purpose}
			The general goals of refactoring are improved software design, improved understanding and help finding bugs.

			There are several scenarios for when refactoring is used. For example, even though a piece of code, class or function has been written with a good design, as the system evolves and new functionality is added the good design might become obsolete. In order to improve the general design, refactoring the original code is needed.

			In another scenario, the code might be bad to begin with. Urgency or lack of time might have forced a developer to solve a problem quickly, without having the time to create a well-designed solution.

			A third scenario is the one for this thesis, refactoring in order to be able to add automated tests for the system.

			\subsubsection{Advantages}
			Software design is improved by reducing complexity. Some techniques include removing
			duplication, breaking up large functions into smaller ones, breaking dependencies, and so on.
			Reducing complexity of the software makes it easier to understand, simpler to maintain and less prone to involve bugs.

			Improving understanding is important since there are more users of the code than one might initially think of. Besides the computer which reads and processes the code, there are the developers who will be reading the code after its been written. As a developer when a new feature is to be added or a bug is to be fixed, a lot of time is spent reading and trying to understand old code.

		(Why is it good to break dependencies, why is separation of concerns good? Do I need to add a part about design that increases testability, or do I keep it here?)
		IIn Refactoring for changeability: a way to go? and Measuring Technology Effects on Software Change Cost, it is shown in a case study that refactoring can significantly decrease the number of customer reported defects and effort needed to make changes.


	\section{Approach}
		\subsection{Comparison criteria}
			\subsubsection{Time}
			One important aspect of this thesis is to compare the two methods with regards to how difficult it is to maintain the tests. An obvious way of doing this is to measure the time it takes to update the tests after the system has changed. 

			Time measurement can be tricky to perform and during the short time span available for the thesis, it is not possible to get enough data in order to do a statistical analysis on it. 

			However, it is still interesting to use the recorded times when discussing the results. The time that is recorded is the time spent working on the problem at hand. Time used for example for discussions and solving problems with the test or development environment will not be recorded as efficient time spent on the test or refactoring effort.
			
			Even though a lot of focus is placed on the maintenance effort for the tests, the time it takes to set up the tests will also be recorded. If one method is very costly it, this needs to be taken into account when comparing the two. As a matter of fact, if both methods are costly, it will still be interesting to know this.

			\subsubsection{Changes}
			There are other ways of measuring effort as well. In {\em A Case study on Regression Test Suite Maintenance in System Evolution} \cite{regression}, the authors log the steps of their work during its progress. 
			Their work builds on top of a previous case study where a system has been added a feature using three different change strategies. Skoglund and Runeson look at the maintenance effort required to update the test suite of the system in each case. 

			The log includes information such as compilation errors in the test
			suite, number of lines of modified test code and number of lines of modified system code. Inspired by this approach, this thesis will also present some of the same metrics.

			\subsubsection{What else?}

		\subsection{Procedure}
		In order to evaluate the benefit of one method over the other, on top pf adding automated tests to this system, it is also necessary to look at how difficult it is to maintain the tests when the system changes. 

		As it is not in the scope of this thesis to implement changes in the system, changes that are planed or have been performed and that are suitable for this research will be found. By looking at the work items that the OMS team has been working on, it is possible to select a change that seems appropriate. 

		Because the source code is under version control, it is possible to select a certain change set from the source code repository and set the system up as it was at a certain point in time.
		The same procedure is done for both the GUI tests and the refactored tests.

		The following procedure will be used:

		\begin{enumerate}
			\item  Tests are created for the scenario prior to the change being implemented. The purpose is to pretend that there were test cases for the system before the change specified in the work item is implemented.
			\item The change is rolled onto the system.
			\item The test cases are changed in response to the change. Tests might be added, modified or removed.
		\end{enumerate}

		\subsection{Case selection}
		With the time constraints for the Master's Thesis, it is necessary to pick a representative set of changes to look at.
		An analysis of the three previous releases and an interview with one of the developers gives an indication of what kind of work items that are typical for this system and this module in particular.

		\subsubsection{Past releases}
		
		\begin{tabular}{ c | c c c | c}
					 	& GUI & Logic & Both & Total\\
			February 	& 1 	& 2 	& 2 	& 5\\
			May 		& 7 	& 1 	& 3 	& 11\\
			September 	& 1 	& 0 	& 1 	& 2\\
			\hline
			Total 		& 9 	& 3 	& 6 	& 18\\
		\end{tabular}


		Two thirds of the work items have impact on the GUI. More than half of these are mere modifications of the GUI, whereas the rest also concern the business logic. Work items that only affect the business logic are few.

		Most work items are of moderate size and a few of the logic changes are a bit larger.
		
		\subsubsection{Future releases}
		The interview reveals that more GUI changes are to be expected in the future as the GUI is being polished and the system owner stresses the need for increased usability. H

		However we don't know for sure how the system will evolve in the future.

		\subsection{Frameworks}

			Two different frameworks are used in this thesis. For GUI tests, the White framework and for unit tests, NUnit. 

			\subsubsection{Project White}
			For this purpose, an open source Framework for .NET applications called White is used. The framework is based on Microsoft's UIAutomation library and window messages. The UIAutomation library was originally developed for Assistive Technology products. That is, products designed to provide better access for individuals with physical or cognitive difficulties, impairments, or disabilities. \cite{white}

			Through this API it is possible to interact with user interface (UI) elements. 
			White provides functions that can be used to search for, access and manipulate UI elements in an application by name, automation id or type. 

			For example:

			\begin{lstlisting}[caption=A method which finds a button with id {\em btnServiceObjects} and clicks on it.]
		Private Sub Click_add_serviceobjects_in(
									ByVal workorder_window As UIItemContainer)
			Dim addButton = workorder_window.Get(OfButton)(SearchCriteria.
												ByAutomationId("btnServiceObjects"))
			addButton.Click()
		End Sub
			\end{lstlisting}
			
			The framework also provides the possibility to log the UI elements found in a certain window, enabling the developer or tester to get further information about their names, types and attributes.

			Since the elements are accessed with code and not through a recording, the tests are less sensitive to changes in the GUI. For example, as a long as a button keeps the same automation id, it doesn't matter if it is moved to another place on the GUI. 

			SOURCE FOR THIS BEING GOOD!!!


			\subsubsection{NUnit}
			(For unit)
			Write about mocks/stubs?

	\section{Case studies}
		\subsection{Case 1: The Service Object List}
		In the Workshop module of the OMS system, focus is on keeping the vehicles in good shape by performing repairs and making sure that they receive servicing in due time. For this, a list of all the objects on the vehicles that can require servicing is being maintained in the database. 

		For each service object has a certain interval between services. This can for example be after a certain amount of kilometers driven or after an amount of time. 

		Some objects are external objects and service is performed on them by an external contractor, the other objects are serviced in the Nobina workshops. When an object is up for service it is said that it has a warning.

		When a mechanic opens the system, a list of all vehicles at this garage that require service is displayed.

		Fig x. Part of the graphical interface of the OMS system. Here a list jobs.

			\subsubsection{Original scenario}
			When opening a work order, all the service objects should be displayed 

			By opening one of those work orders, there is a list with check-boxes in the interface that displays all service objects. The ones that are eligible for service are checked.

			Fig x. Work order, in the middle on the right is the list of service objects.

			\subsubsection{Updated scenario}
			The updated scenarios: When opening a work order, only the service objects eligible should be displayed and In the service object management window, all the service objects should be displayed
			
			This is a new implementation of how the service objects are being presented and the feedback received says that it is hard to visualize which items should receive service since the list is quite long and the box rather small. The development team came up with the idea of showing only the eligible and external items with warnings in the list. The complete list of service object can be accessed by a button which opens up a new interface.

			Fig x. Service object list management window.

			In this interface, it is possible to select other service items to be added to the list that is displayed in the work order.

			When the mechanic closes the work order, the system makes the conclusion that the checked items in the list have received service and updates the database accordingly.


			\subsubsection{Method one}
			\subsubsection{Method two}


		\subsection{Case 2: The Fuel Outtake and Odometer}
			\subsubsection{Method one}
			\subsubsection{Method two}

	\section{Results}
		\subsection{Time}
		\subsection{Effort}
		\subsection{Difficulties - lessons learned}
			\subsubsection{Test environment}

			Confirms the studies in Testing - Prerequisites?

			For the OMS system, there is a test database that is a copy from the real database. Every other day, a fresh copy is made. All the developers share the same database. This presents a problem for this thesis. 

			Since it is not possible to know if some data will be present in the future, it is not certain that the tests written today will still work tomorrow. In some cases, it is certain that they won't work. Why?

			Because of the complex structure of the database it is not possible to create specific test data that can be used for testing.

			If the test requires a bus that has been driven a certain amount of kilometers, it is not possible to simply create such a bus. Neither is it possible to change the amount of kilometers for an existing bus because it might violate existing rules in the database. 

			Also, some operation can only be performed once. For each time a test is rerun, a new object to perform the same action on has to be found.

			Because of these problems, some workarounds have been introduced for the purpose of completing the thesis. In another scenario, I would recommend to solve this issue in a more permanent manner. The workarounds are explained in the different cases.

			This is a difficulty for the GUI tests but it does not impact the unit tests.
		\subsection{Risks}
		I know what happens and might optimize my code for that scenario.

	\section{Further work}
	- analysis of what kind of bugs to the methods catch
	- further analysis of maintainability of longer periods of time 
	- introduction of the Page Object Pattern for more stable gui tests

	\section{Conclusion}
		A separate test environment for gui testing

\begin{thebibliography}{1}
\bibitem{refactor} Martin Fowler {\em Refactoring - Improving the design of everyday code} 1999: Addison-Wesley Professional
\bibitem{regression} Skoglund and Runeson {\em A Case Study on Regression Test Suite Maintenance in System Evolution} 2004 
\bibitem{white} {\em Project White} https://github.com/TestStack/White
\end{thebibliography}

\end{document}

